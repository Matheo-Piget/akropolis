\documentclass{article}

% Language setting
\usepackage[french]{babel}

% Set page size and margins
\usepackage[letterpaper,top=2cm,bottom=2cm,left=3cm,right=3cm,marginparwidth=1.75cm]{geometry}

% Useful packages
\usepackage{amsmath}
\usepackage{graphicx}
\usepackage[colorlinks=true, allcolors=blue]{hyperref}

\title{PG4 : Akropolis}
\author{Nidhal Moussa \& Mathéo Piget \& Benzerdjeb Reyene \& Gbaguidi Nerval \& Chetouani Bilal
}
\date{\today}

\begin{document}

    \maketitle

    \tableofcontents

    \newpage

    \section{Introduction}\label{sec:introduction}

    \subsection{Présentation du Projet}\label{subsec:presentation-du-projet}
    Le projet consiste en la réalisation d'un jeu de société nommé Akrapolis, qui est un jeu tour par tour dont le but est de construire une cité antique.
    Le jeu se déroule sur un plateau de jeu composé de cases hexagonales.
    Chaque joueur possède un certain nombre de ressources qu'il peut utiliser pour construire des bâtiments sur les cases du plateau.
    Chaque bâtiment rapporte des points de victoire à son propriétaire.
    Le joueur qui a le plus de points de victoire à la fin de la partie est déclaré vainqueur.

    \newline
    Réalisé en java et en utilisant la bibliothèque graphique Swing, le jeu est jouable en local sur un ordinateur.

    \subsection{Objectifs}\label{subsec:obectifs}

    L'objectif principal du projet est de réaliser un jeu de société complet et fonctionnel, avec une interface graphique permettant de jouer en local sur un ordinateur.
    Les objectifs secondaires ont été de réaliser un jeu agréable à jouer, avec une interface graphique intuitive et des graphismes de qualité.

    \section{Gestion du projet}\label{sec:gestion-du-projet}

    \subsection{Méthode}\label{subsec:methode}

    \subsection{Répartition des taches}\label{subsec:repartition-des-taches}


    \subsection{Architecture du Projet}\label{subsec:architecture-du-projet}

    \section{Développement et Fonctionnalités}\label{sec:developpement-et-fonctionnalites}

    \subsection{Organisation du Code}\label{subsec:organisation-du-code}

    \subsection{Fonctionnalités Principales}\label{subsec:fonctionnalites-principales}

    \subsubsection{Mécaniques de Jeu}

    \subsubsection*{Interactions Utilisateur}

    \subsection{Gestion des Ressources}\label{subsec:gestion-des-ressources}

    \subsubsection*{Images et Textures}

    \begin{itemize}
        \item \textbf{Chargement des Images :} Les images et textures du jeu sont stockées dans des fichiers séparés.

        \item Le jeu les charge dynamiquement lors de son exécution.

        \item \textbf{Format des Images :} Les images sont au format PNG pour assurer une qualité graphique optimale tout en minimisant la taille des fichiers.
    \end{itemize}

    \subsubsection*{Fichiers Sonores}

    \begin{itemize}
        \item \textbf{Musique de Fond :} La musique de fond du jeu est également gérée en tant que ressource sonore, contribuant à l'ambiance globale.
    \end{itemize}

    \subsection{Fichiers de Configuration}\label{subsec:fichiers-de-configuration}


    \subsection{Logique de Jeu}\label{subsec:logique-de-jeu}


    \section{Difficultés Rencontrées}\label{sec:difficultes-rencontrees}
    \newpage
    \section{Conclusion}\label{sec:conclusion}




\end{document}
